\documentclass{beamer}
\usepackage{hyperref, graphicx}
\usepackage{amsmath,amsfonts, amssymb, physics}
\usepackage{amsthm}
\usepackage[english]{babel}
\usepackage{enumitem}

\newcommand{\rn}{\mathbb{R}}
\newcommand{\nn}{\mathbb{N}}
\newcommand{\q}{\mathbb{Q}}
\newcommand{\p}{\mathcal{P}}
\newcommand{\bigO}{\mathcal{O}}
\newcommand{\z}{\mathbb{Z}}
\newcommand{\lub}{\text{lub}}
%Information to be included in the title page:
\title[Runge-Kutta Methods to Solve ODEs] %optional
{Runge-Kutta Methods to Solve ODEs}
\usetheme{Madrid}
\subtitle{Instructor: Koel Das}

\author[Sarkar, Sarkar] % (optional, for multiple authors)
{Debayan~Sarkar\inst{1} \and Diptanuj~Sarkar\inst{2}}

\institute % (optional)
{
  \inst{1}%
  Dept. of Physical Sciences\\
  IISER Kolkata
  \and
  \inst{2}%
  Dept. of Physical Sciences\\
  IISER Kolkata
}

\date[\today] % (optional)
{MA3105: Numerical Analysis Final Project}
\begin{document}

\frame{\titlepage}
\begin{frame}
\frametitle{Introduction}
        Ordinary Differential Equations (ODEs) are fundamental in modeling real-world phenomena across various fields of science and engineering. While analytical solutions are ideal, they are often unattainable for complex equations. Numerical methods, such as the Forward Euler Method and Runge-Kutta Methods, provide practical alternatives. This report focuses on the Runge-Kutta methods, specifically the second and fourth-order schemes, and compares them to the Forward Euler Method.
\end{frame}
\begin{frame}
    \frametitle{Numerical Methods to Solve ODEs}
    In this project we have looked at three different ODE solvers. They have been listed below:
    \begin{itemize}
        \item Forward Euler Method
        \item Runge-Kutta Method of the Second Order (RK2)
        \item Runge-Kutta Method of the Fourth Order (RK4)
    \end{itemize}
   In the next slides we go over these methods in more details, beginning with the most basic one. 
\end{frame}
\begin{frame}
    \frametitle{Euler Method}
    This is the simplestway to solve ODEs numerically. This is a first order method. As in, the error in it grows as the square of the step size. The motivation for it has been shown below.

\end{frame}
\begin{frame}
    \frametitle{Euler Method}

    Let $y(x)$ be a smooth function. Then, from its taylor expansion we have
    \begin{align*}
        &y(x + h) = y(x) + y'(x)h + \sum_{n = 2}^{\infty}y^{(n)}(x)h^n \\
        \Rightarrow &y(x + h) = y(x) + y'(x)h + \mathcal{O}(h^2) \\
        \Rightarrow &y(x + h) = y(x) + y'(x)h \tag{Ignoring the second order terms}
    \end{align*}
    This gives us the expression for iteratively calculating $y(x)$ using the forward Euler Method:
    $$y(x + h) = y(x) + hf(x, y)$$
    Denoting them in terms of $x_n$ and $y_n$ we get the following expression:
    $$\boxed{y_{n+1} = y_n + hf(x_n, y_n)}$$
\end{frame}
\begin{frame}
    \frametitle{Runge-Kutta Methods}
Runge-Kutta methods are a family of iterative methods to solve ODEs which were first introduced in the 1900s. These methods improve upon the Forward Euler Method by considering intermediate slopes to achieve higher accuracy. The Runge-Kutta methods implemented here have been discussed in further detail below.
\end{frame}
\begin{frame}
    \frametitle{Runge-Kutta Methods: Second Order (Midpoint Method)}

    The most commonly used second order Runge-Kutta method is the midpoint method, where we improve upon the Euler Method using the slope of the midpoint. The algorithm goes as follows. Given a step-size of $h > 0$, in each iteration we define the following:
    \begin{align*}
        k_1 &= f(x_n , y_n) \\
        k_2 &= f\left(x_n + \frac{h}{2}, y_n + \frac{h}{2}k_1\right)
    \end{align*}

Now using these values we calculate the next term in the iteration using the following expression:

$$y_{n + 1} = y_n + hk_2$$

Substituting the value of $k_2$ in this expression we get:

$$\boxed{y_{n + 1} = y_n + hf\left(x_n + \frac{h}{2}, y_n + \frac{h}{2}f(x_n , y_n)\right)}$$
\end{frame}
\begin{frame}
    \frametitle{Runge-Kutta Methods: Second Order}
    However this is not the only second order Runge-Kutta Method for solving an IVP. Any second order 
Runge-Kutta Method can be parameterised by $\alpha$ and given by the formula:

$$\boxed{y_{n+1} = y_n + h\left[\left(1 - \cfrac{1}{2\alpha}\right)f(x_n, y_n) + \cfrac{1}{2\alpha}f(x_n + \alpha h, y_n + \alpha hf(x_n, y_n))\right]}$$
    This formula has been derived in the next few slides.
\end{frame}
\begin{frame}
    \frametitle{Runge-Kutta Methods: Second Order}
    First let us define the following terms:
    \begin{align*}
        &k_1 = hf(x_n, y_n) \\
        &k_2 = hf(x_n + \alpha h, y_n + \beta k_1) \\
        &y_{n+1} = y_n + ak_1 + bk_2 \tag{1}
    \end{align*}
    Where, $a$, $b$, $\alpha$, $\beta$ are constants.
    
\end{frame}
\begin{frame}
    \frametitle{Runge-Kutta Methods: Second Order}
    Now, consider the taylor series expansion of $y(x)$ upto the second order.
    \begin{align*}
        &y(x + h) = y(x) + h\dv{y}{x} + \cfrac{h^2}{2} \dv[2]{y}{x} + \bigO(h^3) \\
        \Rightarrow &y(x + h) = y(x) + hf(x, y) + \cfrac{h^2}{2}\left(\pdv{f}{x} + f(x, y)\pdv{f}{y}\right) + \bigO(h^3)\\ 
        \Rightarrow &y_{n+1} = y_n + hf(x_n, y_n) + \cfrac{h^2}{2}\left(\pdv{f}{x} + f(x_n, y_n)\pdv{f}{y}\right) + \bigO(h^3) \tag{2}\\ 
    \end{align*}
\end{frame}
\begin{frame}
    \frametitle{Runge-Kutta Methods: Second Order}
    We can also do a taylor series expansion for $k_2$ upto an error of $\bigO(h^3)$.
    \begin{align*}
        k_2 &= hf(x_n + \alpha h, y_n + \beta k_1) \\ 
            &= h\left[ f(x_n, y_n) + \pdv{f}{x}\alpha h + \pdv{f}{y}\beta k_1 \right] + \bigO(h^3) \\ 
    \end{align*}
\end{frame}
\begin{frame}
    \frametitle{Runge-Kutta Methods: Second Order}
    We can also do a taylor series expansion for $k_2$ upto an error of $\bigO(h^3)$.
    \begin{align*}
        k_2 &= hf(x_n + \alpha h, y_n + \beta k_1) \\ 
            &= h\left[ f(x_n, y_n) + \pdv{f}{x}\alpha h + \pdv{f}{y}\beta k_1 \right] + \bigO(h^3) \\ 
    \end{align*}
    Substituting this value of $k_2$ into (1) we get, 
    \begin{align*}
        y_{n+1} = y_n + h(a + b)f(x_n, y_n) + bh^2\left[ \alpha \pdv{f}{x} + \beta f(x_n, y_n) \pdv{f}{y} \right] + \bigO(h^3) \tag{3}
    \end{align*}
\end{frame}
\begin{frame}
    \frametitle{Runge-Kutta Methods: Second Order}
    Comparing (2) and (3) and equating the coefficients we get the following equations.
    $$\alpha b = \cfrac{1}{2} \Rightarrow b = \cfrac{1}{2\alpha}$$
    $$\beta b = \cfrac{1}{2} \Rightarrow \beta = \alpha$$
    $$a + b = 1 \Rightarrow a = 1 - \cfrac{1}{2\alpha}$$
\end{frame}
\begin{frame}
    \frametitle{Runge-Kutta Methods: Second Order}
    Then (1) can be re-written parameterised only in terms of $\alpha$ as,
    $$\boxed{y_{n+1} = y_n + h\left[\left(1 - \cfrac{1}{2\alpha}\right)f(x_n, y_n) + \cfrac{1}{2\alpha}f(x_n + \alpha h, y_n + \alpha hf(x_n, y_n))\right]}$$

    As is clear from this derivation, this method is valid upto an error of the order of $\bigO(h^3)$.
\end{frame}
\begin{frame}
    \frametitle{Runge-Kutta Methods: Fourth Order}
    The fourth order Runge-Kutta Method, also known as the classic Runge-Kutta Method is the most widely known member of the Runge-Kutta Family, and is known as 'RK4'. The algorithm goes as follows. Given a step-size of $h > 0$, in each iteration we define the following:
    \begin{align*}
        k_1 &= f(x_n, y_n) \\    
        k_2 &= f\left(x_n + \frac{h}{2}, y_n + \frac{h}{2}k_1\right) \\
        k_3 &= f\left(x_n + \frac{h}{2}, y_n + \frac{h}{2}k_2\right) \\
        k_4 &= f\left(x_n + h, y_n + hk_3\right) \\
    \end{align*}
\end{frame}
\begin{frame}
    \frametitle{Runge-Kutta Methods: Fourth Order}
    Now, using these intermediate slopes, we calculate the next term in the iteration using the following expression:
    $$\boxed{y_{n + 1} = y_n + \frac{h}{6}(k_1 + 2k_2 + 2k_3 + k_4)}$$
A derivation similar to that of RK2 can be done for this as well and it can be shown that RK4 is a fourth order method i.e. the error in the numerically obtained solution from the analytical solution is of the order of $\mathcal{O}(h^5)$.
\end{frame}
\begin{frame}
    \frametitle{Given ODEs}
    We were asked to solve the following differential equations using the aforementioned ODE solvers.
    $$\frac{dy}{dx} = y - x$$
    $$\frac{dy}{dx} = y - x^2$$   
    The initial values provided for both these equations is $y(0) = \cfrac{2}{3}$
\end{frame}
\begin{frame}
    \frametitle{Direction Fields}
    \begin{figure}[h]
        \centering
        \includegraphics[width=\linewidth]{"direction_fields.png"}
        \caption{Direction Fields for the given ODEs}
    \end{figure}
\end{frame}
\begin{frame}
    \frametitle{Analytical Solutions}
    The analytical solutions to these equations can be computed by converting these into exact differentials using the integrating factor $\mu(x) = e^{-x}$. Solving these equations with the intial condition $y(0)$ gives us the solutions
    $$\boxed{y_1(x) = 1 + x + (y(0) - 1)e^x}$$
    $$\boxed{y_2(x) = 2 + 2x + x^2 + (y(0) - 2)e^x}$$
\end{frame}

\begin{frame}
    \frametitle{Solving the ODEs}
    The given IVPs were solved using the priorly mentioned numerical methods, implemented in python.
    A plot of the solutions obtained on the interval $x \in [0, 5]$ has been included below.
\end{frame}

\begin{frame}
    \frametitle{Solving the ODEs}
    \begin{figure}[h]
        \centering
        \includegraphics[width=\linewidth]{"solutions.png"}
        \caption{Plot of the solutions of the ODEs using the priorly mentioned numerical methods}
    \end{figure}
\end{frame}

\begin{frame}
    \frametitle{Solving the ODEs: Comparision and Analysis}
    Looking at the plots, we can compare the methods and conclude the following:
    \begin{itemize}
        \item \textbf{Forward Euler Method:} Produced noticeable deviations from the exact solution, and as the value of $x$ grew, the
            deviation became larger.
        \item \textbf{RK2 Method:} Significantly reduced errors compared to Forward Euler. The inclusion of intermediate slopes provided better stability and accuracy.
        \item \textbf{RK4 Method:} Delivered solutions nearly indistinguishable from the exact solution. The higher accuracy of $\bigO(h^5)$ made RK4 the most reliable among the three methods for the given ODEs. 
    \end{itemize}
\end{frame}
\begin{frame}
    \frametitle{Solving the ODEs: Varying the Initial Values} 
    We varied the initial value $y(0)$ and plotted the analytical solution and RK4 solutions to see how they compare. 
    The step-size taken for the solutions was $h = 0.025$.
\end{frame}
\begin{frame}
    \frametitle{Solving the ODEs: Varying the Initial Values} 
    \begin{figure}[h]
        \centering
        \includegraphics[width=\linewidth]{"varied_initial_vals_RK4.png"}
        \caption{Plots of the Solutions of the given Differential Equations with varying values of $y(0)$}
    \end{figure}
\end{frame}
\end{document}
